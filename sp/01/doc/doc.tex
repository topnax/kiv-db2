\documentclass[12pt, a4paper]{article}

\usepackage[czech]{babel}
\usepackage{lmodern}
\usepackage[utf8]{inputenc}
\usepackage[T1]{fontenc}
\usepackage[pdftex]{graphicx}
\usepackage{amsmath}
\usepackage[hidelinks,unicode]{hyperref}
\usepackage{float}
\usepackage{listings}
\usepackage{tikz}
\usepackage{xcolor}
\usepackage{tabularx}
\usepackage[final]{pdfpages}
\usepackage{syntax}


\definecolor{mauve}{rgb}{0.58,0,0.82}
\usetikzlibrary{shapes,positioning,matrix,arrows}

\newcommand{\img}[1]{(viz obr. \ref{#1})}

\definecolor{pblue}{rgb}{0.13,0.13,1}
\definecolor{pgreen}{rgb}{0,0.5,0}
\definecolor{pred}{rgb}{0.9,0,0}
\definecolor{pgrey}{rgb}{0.46,0.45,0.48}


\lstdefinestyle{flex}{
    frame=tb,
    aboveskip=3mm,
    belowskip=3mm,
    showstringspaces=false,
    columns=flexible,
    basicstyle={\small\ttfamily},
    numbers=none,
    numberstyle=\tiny\color{black},
    keywordstyle=\color{black},
    commentstyle=\color{black},
    stringstyle=\color{black},
    breaklines=true,
    breakatwhitespace=true,
    tabsize=3
}

\lstset{
    frame=tb,
    language=XML,
    aboveskip=3mm,
    belowskip=3mm,
    showstringspaces=false,
    columns=flexible,
    basicstyle={\small\ttfamily},
    numbers=none,
    numberstyle=\tiny\color{gray},
    keywordstyle=\color{blue},
    commentstyle=\color{pgreen},
    stringstyle=\color{mauve},
    breaklines=true,
    breakatwhitespace=true,
    tabsize=3
}


\let\oldsection\section
\renewcommand\section{\clearpage\oldsection}

\begin{document}
	% this has to be placed here, after document has been created
	% \counterwithout{lstlisting}{chapter}
	\renewcommand{\lstlistingname}{Ukázka kódu}
	\renewcommand{\lstlistlistingname}{Seznam ukázek kódu}
    \begin{titlepage}

        \centering

        \vspace*{\baselineskip}
        \begin{figure}[H]
        \centering
        \includegraphics[width=7cm]{img/fav-logo.jpg}
        \end{figure}

        \vspace*{1\baselineskip}

        \vspace{0.75\baselineskip}

        \vspace{0.5\baselineskip}
        {Semestrální práce z předmětu KIV/DB2}

        {\LARGE\sc Databázová technologie XML (Oracle)\\}

        \vspace{4\baselineskip}

        \vspace{0.5\baselineskip}

        {\sc\Large Stanislav Král \\}
        \vspace{0.5\baselineskip}
        {A20N0091P}

        \vfill

        {\sc Západočeská univerzita v Plzni\\
        Fakulta aplikovaných věd}

    \end{titlepage}


    % TOC
    \tableofcontents
    \pagebreak

    
\section{Popis řešeného tématu}
Zvolené téma představuje situaci investora, který nakupuje a prodává kryptoměny na různých směnárnách či burzách. Předpokladem tohoto tématu je to, že služby realizující samotný nákup či prodej kryptoměn umožňují export všech uskutečněných obchodů ve formátu XML se specifickým schématem. 

Pro takového investora je důležité, aby všechny informace o nákupech byly jednoduše dohledatelné a aby jednotlivé kryptoměny mohl třídit do portfólií. Tento investor by si mohl například chtít třídit nákupy do portfólia, ve kterém si bude vést obchody s významnějšími kryptoměnami (např. Bitcoin či Ethereum) a obchody s méně významnými kryptoměnami (dle tržní kapitalizace např. Basic Attention Token či Solana) do jiného portfólia.
    
Dále je také nutné, aby ke každému obchodu byla přiřazena informace v jaké měně byl uskutečněn. Toto se nejčastěji reprezentuje pomocí tzv. \textit{obchodních párů}, které se skládají z měny (např. USD) a dané komodity (např. BTC).

Pomocí takto popsaného nástroje (databáze), který tento problém řeší, si investor může např. svá portfólia porovnat oproti aktuálnímu kurzu kryptoměn, a zjistit jak se jeho obchodům daří. Takový nástroj by také našel využití i při vyplňování daňového přiznání.

\section{Popis datové struktury}

V databázi jsou definovány obchodní páry skládající se ze jména komodity, zkratky komodity a měně, ve které se buda daná komodita obchodovat. Dále se v databázi nachází seznam portfólií, které jsou popsány identifikátorem, názvem a popisem. V těchto portfóliích se dále mohou vyskytovat jednotlivé obchody, které nesou základní informace o uskutečněném nákupu a odkazují se na obchodní pár.

\subsection{Obchodní páry}
Seznam obchodních páru se vyskytuje v prvku \texttt{trading-pairs}. Jednotlivé obchodní páry jsou reprezentovány prvkami \texttt{trading-pair} a v atributu \texttt{id} se nachází jejich identifikátory. V tomto prvku se pak dále vyskytují následující prvky popisující obchodní pár:

\begin{itemize}
    \item \texttt{name} - název kryptoměny v obchodním páru
    \item \texttt{symbol} - zkratka názvu kryptoměny v obchodním páru 
    \item \texttt{currency} - zkratka měny v obchodním páru 
\end{itemize}


\begin{lstlisting}
<trading-pairs>
    <trading-pair id="1">
        <name>Bitcoin</name>
        <symbol>BTC</symbol>
        <currency>USD</currency>
    </trading-pair>
    <trading-pair id="2">
        <name>Litecoin</name>
        <symbol>LTC</symbol>
        <currency>EUR</currency>
    </trading-pair>
</trading-pairs>
\end{lstlisting}

\subsection{Portfólia}
Seznam portfólií se vyskytuje v prvku \texttt{portfolios}. Jednotlivá portfólia jsou reprezentována prvkami \texttt{portfolio} a v atributu \texttt{id} se nachází jejich identifikátory. Tyto prvky dále obsahují následující atributy:

\begin{itemize}
    \item \texttt{name} - pojmenování portfólia
    \item \texttt{description} - vlastní popis portfólia
\end{itemize}

Pokud se v daném portfóliu vyskytují nějaké obchody, tak jsou reprezentovány prvkami \texttt{order} ve vnořeném prvku \texttt{orders}.

Důležité je zmínit, že napříč více XML soubory v databázi se může vyskytovat prvek popisující jedno portfólio i více než jednou. Toto je způsobené tím, že jednotlivé nákupy portfólia se nachází v různých XML souborech. Avšak název a popis jednoho portfólia by se měl vyskytovat v XML souborech pouze jednou.

\begin{lstlisting}
<portfolios>
    <portfolio id="1" name="Main Portfolio" description="Tracking Bitcoin/Ethereum transactions">
        <orders>
            ...
        </orders>
    </portfolio>
    <portfolio id="2" name="Altcoin Portfolio" description="Tracking altcoin portfolio">
        <orders>
            ...
        </orders>
    </portfolio>
    <portfolio id="3" name="Parents portfolio" description="Tracking investments of my parents"/>
</portfolios>

\end{lstlisting}


\end{document} 
