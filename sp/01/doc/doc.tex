\documentclass[12pt, a4paper]{article}

\usepackage[czech]{babel}
\usepackage{lmodern}
\usepackage[utf8]{inputenc}
\usepackage[T1]{fontenc}
\usepackage[pdftex]{graphicx}
\usepackage{amsmath}
\usepackage[hidelinks,unicode]{hyperref}
\usepackage{float}
\usepackage{listings}
\usepackage{tikz}
\usepackage{xcolor}
\usepackage{tabularx}
\usepackage[final]{pdfpages}
\usepackage{syntax}


\definecolor{mauve}{rgb}{0.58,0,0.82}
\usetikzlibrary{shapes,positioning,matrix,arrows}

\newcommand{\img}[1]{(viz obr. \ref{#1})}

\definecolor{pblue}{rgb}{0.13,0.13,1}
\definecolor{pgreen}{rgb}{0,0.5,0}
\definecolor{pred}{rgb}{0.9,0,0}
\definecolor{pgrey}{rgb}{0.46,0.45,0.48}


\lstdefinestyle{flex}{
    frame=tb,
    aboveskip=3mm,
    belowskip=3mm,
    showstringspaces=false,
    columns=flexible,
    basicstyle={\small\ttfamily},
    numbers=none,
    numberstyle=\tiny\color{black},
    keywordstyle=\color{black},
    commentstyle=\color{black},
    stringstyle=\color{black},
    breaklines=true,
    breakatwhitespace=true,
    tabsize=3
}

\lstset{
    frame=tb,
    language=C,
    aboveskip=3mm,
    belowskip=3mm,
    showstringspaces=false,
    columns=flexible,
    basicstyle={\small\ttfamily},
    numbers=none,
    numberstyle=\tiny\color{gray},
    keywordstyle=\color{blue},
    commentstyle=\color{pgreen},
    stringstyle=\color{mauve},
    breaklines=true,
    breakatwhitespace=true,
    tabsize=3
}


\let\oldsection\section
\renewcommand\section{\clearpage\oldsection}

\begin{document}
	% this has to be placed here, after document has been created
	% \counterwithout{lstlisting}{chapter}
	\renewcommand{\lstlistingname}{Ukázka kódu}
	\renewcommand{\lstlistlistingname}{Seznam ukázek kódu}
    \begin{titlepage}

        \centering

        \vspace*{\baselineskip}
        \begin{figure}[H]
        \centering
        \includegraphics[width=7cm]{img/fav-logo.jpg}
        \end{figure}

        \vspace*{1\baselineskip}

        \vspace{0.75\baselineskip}

        \vspace{0.5\baselineskip}
        {Semestrální práce z předmětu KIV/DB2}

        {\LARGE\sc Databázová technologie XML (Oracle)\\}

        \vspace{4\baselineskip}

        \vspace{0.5\baselineskip}

        {\sc\Large Stanislav Král \\}
        \vspace{0.5\baselineskip}
        {A20N0091P}

        \vfill

        {\sc Západočeská univerzita v Plzni\\
        Fakulta aplikovaných věd}

    \end{titlepage}


    % TOC
    \tableofcontents
    \pagebreak

    
\section{Popis řešeného tématu}
Zvolené téma představuje situaci investora, který nakupuje a prodává kryptoměny na různých směnárnách či burzách. Předpokladem tohoto tématu je to, že služby realizující samotný nákup či prodej kryptoměn umožňují export všech uskutečněných obchodů ve formátu XML se specifickým schématem. 

Pro takového investora je důležité, aby všechny informace o nákupech byly jednoduše dohledatelné a aby jednotlivé kryptoměny mohl třídit do portfólií. Tento investor by si mohl například chtít třídit nákupy do portfólia, ve kterém si bude vést obchody s významnějšími kryptoměnami (např. Bitcoin či Ethereum) a obchody s méně významnými kryptoměnami (dle tržní kapitalizace např. Basic Attention Token či Solana) do jiného portfólia.
    
Dále je také nutné, aby ke každému obchodu byla přiřazena informace v jaké měně byl uskutečněn. Toto se nejčastěji reprezentuje pomocí tzv. \textit{obchodních párů}, které se skládají z měny (např. USD) a dané komodity (např. BTC).

Pomocí takto popsaného nástroje (databáze), který tento problém řeší, si investor může např. svá portfólia porovnat oproti aktuálnímu kurzu kryptoměn, a zjistit jak se jeho obchodům daří. Takový nástroj by také našel využití i při vyplňování daňového přiznání.
\end{document}    
