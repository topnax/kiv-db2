\documentclass[12pt, a4paper]{article}

\usepackage[czech]{babel}
\usepackage{lmodern}
\usepackage[utf8]{inputenc}
\usepackage[T1]{fontenc}
\usepackage[pdftex]{graphicx}
\usepackage{amsmath}
\usepackage[hidelinks,unicode]{hyperref}
\usepackage{float}
\usepackage{listings}
\usepackage{tikz}
\usepackage{xcolor}
\usepackage{tabularx}
\usepackage[final]{pdfpages}
\usepackage{syntax}


\definecolor{mauve}{rgb}{0.58,0,0.82}
\usetikzlibrary{shapes,positioning,matrix,arrows}

\newcommand{\img}[1]{(viz obr. \ref{#1})}

\definecolor{pblue}{rgb}{0.13,0.13,1}
\definecolor{pgreen}{rgb}{0,0.5,0}
\definecolor{pred}{rgb}{0.9,0,0}
\definecolor{pgrey}{rgb}{0.46,0.45,0.48}

%define Javascript language
\lstdefinelanguage{JavaScript}{
keywords={typeof, new, true, false, catch, function, return, null, catch, switch, var, if, in, while, do, else, case, break},
keywordstyle=\color{blue}\bfseries,
ndkeywords={class, export, boolean, throw, implements, import, this},
ndkeywordstyle=\color{darkgray}\bfseries,
identifierstyle=\color{black},
sensitive=false,
comment=[l]{//},
morecomment=[s]{/*}{*/},
commentstyle=\color{purple}\ttfamily,
stringstyle=\color{pgreen}\ttfamily,
morestring=[b]',
morestring=[b]"
}


\lstdefinelanguage{json}{
    basicstyle=\normalfont\ttfamily,
    commentstyle=\color{eclipseStrings}, % style of comment
    stringstyle=\color{eclipseKeywords}, % style of strings
    numbers=left,
    numberstyle=\scriptsize,
    stepnumber=1,
    numbersep=8pt,
    showstringspaces=false,
    breaklines=true,
    frame=lines,
    string=[s]{"}{"},
    comment=[l]{:\ "},
    morecomment=[l]{:"},
    literate=
        *{0}{{{\color{numb}0}}}{1}
         {1}{{{\color{numb}1}}}{1}
         {2}{{{\color{numb}2}}}{1}
         {3}{{{\color{numb}3}}}{1}
         {4}{{{\color{numb}4}}}{1}
         {5}{{{\color{numb}5}}}{1}
         {6}{{{\color{numb}6}}}{1}
         {7}{{{\color{numb}7}}}{1}
         {8}{{{\color{numb}8}}}{1}
         {9}{{{\color{numb}9}}}{1}
}

\lstset{
    frame=tb,
    language=json,
    aboveskip=3mm,
    belowskip=3mm,
    showstringspaces=false,
    columns=flexible,
    basicstyle={\small\ttfamily},
    numbers=none,
    numberstyle=\tiny\color{gray},
    keywordstyle=\color{blue},
    commentstyle=\color{pgreen},
    stringstyle=\color{mauve},
    breaklines=true,
    breakatwhitespace=true,
    tabsize=3
}


\let\oldsection\section
\renewcommand\section{\clearpage\oldsection}

\begin{document}
	% this has to be placed here, after document has been created
	% \counterwithout{lstlisting}{chapter}
	\renewcommand{\lstlistingname}{Ukázka kódu}
	\renewcommand{\lstlistlistingname}{Seznam ukázek kódu}
    \begin{titlepage}

        \centering

        \vspace*{\baselineskip}
        \begin{figure}[H]
        \centering
        \includegraphics[width=7cm]{img/fav-logo.jpg}
        \end{figure}

        \vspace*{1\baselineskip}

        \vspace{0.75\baselineskip}

        \vspace{0.5\baselineskip}
        {Semestrální práce z předmětu KIV/DB2}

        {\LARGE\sc NOSQL databáze MongoDB\\}

        \vspace{4\baselineskip}

        \vspace{0.5\baselineskip}

        {\sc\Large Stanislav Král \\}
        \vspace{0.5\baselineskip}
        {A20N0091P}

        \vfill

        {\sc Západočeská univerzita v Plzni\\
        Fakulta aplikovaných věd}

    \end{titlepage}


    % TOC
    \tableofcontents
    \pagebreak

    
\section{Popis řešeného tématu}
Zvolené téma představuje situaci investora, který nakupuje a prodává kryptoměny na různých směnárnách či burzách. Předpokladem tohoto tématu je to, že služby realizující samotný nákup či prodej kryptoměn umožňují export všech uskutečněných obchodů ve formátu JSON. 

Pro takového investora je důležité, aby všechny informace o nákupech byly jednoduše dohledatelné a aby jednotlivé kryptoměny mohl třídit do portfólií. Tento investor by si mohl například chtít třídit nákupy do portfólia, ve kterém si bude vést obchody s významnějšími kryptoměnami (např. Bitcoin či Ethereum) a obchody s méně významnými kryptoměnami (dle tržní kapitalizace např. Basic Attention Token či Solana) do jiného portfólia.
    
Dále je také nutné, aby ke každému obchodu byla přiřazena informace v jaké měně byl uskutečněn. Toto se nejčastěji reprezentuje pomocí tzv. \textit{obchodních párů}, které se skládají z měny (např. USD) a dané komodity (např. BTC).

Pomocí takto popsaného nástroje (databáze), který tento problém řeší, si investor může např. svá portfólia porovnat oproti aktuálnímu kurzu kryptoměn, a zjistit jak se jeho obchodům daří. Takový nástroj by také našel využití i při vyplňování daňového přiznání.

\section{Popis datové struktury}

V kolekci \texttt{portfolios} se nachází seznam portfólií, které jsou popsány identifikátorem, názvem a popisem. Každé portfólio ještě obsahuje atribut \texttt{tradingPairs}, který obsahuje seznam obchodních párů, se kterými dané portfólio pracuje. Každý obchodní pár se skládá ze zkratky obchodované komodity a měny, ve které budou s touto komoditou prováděny obchody. 

Kolekce \texttt{transactions} obsahuje jednotlivé obchody, které nesou základní informace o uskutečněném nákupu či prodeji a obsahují obchodní pár, kterého se týkají


\subsection{Portfólia}
Seznam portfólií se vyskytuje v kolekci \texttt{portfolios}. Jednotlivá portfólia jsou reprezentována dokumenty s následující strukturou:
\begin{itemize}
    \item \texttt{\_id} -- identifikátor portfólia
    \item \texttt{name} -- pojmenování portfólia
    \item \texttt{description} -- vlastní popis portfólia
    \item \texttt{tradingPairs} -- seznam obchodních párů, kde každý pár má následující strukturu:
        \begin{itemize}
            \item \texttt{symbol} -- zkratka obchodované komodity
            \item \texttt{currency} -- zkratka měny, ve které bude komodita určená zkratkou \texttt{symbol} obchodována
        \end{itemize}
\end{itemize}

\begin{lstlisting}
{
  "_id": 0,
  "name": "Main Portfolio",
  "description": "Tracking Bitcoin/Ethereum",
  "tradingPairs": [
    {
      "symbol": "btc",
      "currency": "usd"
    },
    {
      "symbol": "eth",
      "currency": "usd"
    }
  ]
}
\end{lstlisting}

Pokud jsou k danému portfóliu přiřazené nějaké obchody, tak se nachází v kolekci \texttt{transactions} a referencují ho pomocí atributu \texttt{\_id}.

\subsection{Uskutečněné obchody}
Seznam obchodů se nachází v kolekci \texttt{orders}. Prvky této jsou kolekce obsahují následující atributy: 
\begin{itemize}
    \item \texttt{\_id} -- identifikátor obchodu
    \item \texttt{portfolioId} -- identifikátor portfólia, do kterého daný obchod patří
    \item \texttt{size} -- množství kryptoměny, která byla obchodována
    \item \texttt{filledPrice} -- cena za jednu minci kryptoměny
    \item \texttt{fee} -- poplatek za uskutečněný nákup
    \item \texttt{timestamp} -- datum a čas, kdy se nákup uskutečnil ve formátu UNIX Timestamp
    \item \texttt{tradingPair} -- obchodní pár, kterého se tento obchod týkal, skládající se z následujících atributů:
        \begin{itemize}
            \item \texttt{symbol} -- zkratka obchodované komodity
            \item \texttt{currency} -- zkratka měny, ve které bude komodita určená zkratkou \texttt{symbol} obchodována
        \end{itemize}
\end{itemize}

Pokud daný obchod představoval prodej, tak prvek kolekce obsahuje atribut \texttt{sell} nastavený na hodnotu \texttt{true}.

\begin{lstlisting}
{
  "_id": 0,
  "portfolioId": 0,
  "size": 5,
  "filledPrice": 601,
  "fee": 1,
  "timestamp": 1568204888,
  "tradingPair": {
    "symbol": "eth",
    "currency": "usd"
  }
}\end{lstlisting}

\section{Popis dotazů nad databázovou strukturou}
Ke zjištení užitečných informací o obchodování bylo vytvořeno 7 dotazů.

\subsection{Vyhledání všech portfólií obchodujících kryptoměnu ADA}
Pro vyhledání všech portfólií, kterých se týká jakýkoliv obchodní pár s kryptoměnou ADA, slouží následující dotaz:

\begin{lstlisting}[language=JavaScript]
db.portfolios.find({"tradingPairs.symbol": "ada"})
\end{lstlisting}
Výsledek obsahuje kolekci dokumentů představujících portfólia.

\subsection{Vyhledání všech BTC transakcí s objemem větší než 0.33 BTC}
Pro vyhledání všech transakcí, kdy se obchodoval obchodní pár BTC/EUR a objem byl větší než 0.33 BTC, slouží následující dotaz:

\begin{lstlisting}[language=JavaScript]
db.transactions.find({\$and:[{"size": {$gte: 0.33}}, {"tradingPair.symbol": "btc"}, {"tradingPair.currency": "eur"}]})

\end{lstlisting}
Výsledek obsahuje kolekci dokumentů představujících uskutečněné transakce.

\subsection{Vyhledání všech prodejů}
K vyhledání všech transakcích, kdy byla prodávána kryptoměna, slouží následující dotaz:

\begin{lstlisting}[language=JavaScript]
db.transactions.find({"sell": {$exists: true}})
\end{lstlisting}
Výsledek obsahuje kolekci dokumentů představujících transakce.


\subsection{Vyhledání všech transakcí s kryptoměnou ADA či DOT}
K vyhledání všech transakcích, kdy byla obchodována kryptoměna ADA či DOT, slouží následující dotaz:

\begin{lstlisting}[language=JavaScript]

db.transactions.aggregate([{$match:{"tradingPair.symbol": {$in: ["ada", "dot"]}}}, {$project: {"size": 1, "filledPrice": 1, "tradingPair": {$concat:["$tradingPair.symbol", "/", "$tradingPair.currency"]}}}])

\end{lstlisting}

Výsledek obsahuje kolekci projekovaných dokumentů představujících transakce, kdy každý dokument obsahuje pouze základní informace o transakci (pole \texttt{size}, \texttt{filledPrice}) a pole \texttt{tradingPair}, které se skládá z textové reprezentace obchodovaného páru.


\subsection{Vyhledání všech portfólií jejichž název odpovídá regulárnímu výrazu}
Pro demonstraci použití vyhledávání portfólií, jejichž název odpovídá nějakému regulárnímu výrazu, slouží následující dotaz:

\begin{lstlisting}[language=JavaScript]
db.portfolios.find({"description": {$regex: "(Tracking )(.*)( of)"}})
\end{lstlisting}

Výsledek je kolekcí všech dokumentů představující portfólia, jejichž název odpovídá danému regulárnímu výrazu. V přiloženém ukázkovém dotaze jsou vyhledána všechna portfólia, která v názvu obsahují slova \textit{Tracking} a \textit{of}, kdy mezi těmito slovy může být libovolný počet znaků.

\subsection{Výpočet sumy všech uskutečněných nákupů kryptoměny Bitcoin ve vybraném portfóliu}
Následující dotaz vypočítá sumu všech uskutečněných nákupů nějaké kryptoměny ve vybraném portfóliu:

\begin{lstlisting}[language=JavaScript]

db.transactions.aggregate([{$match: {$and:[{"portfolioId":2}, {"tradingPair.symbol": "btc"}, {"sell": {$exists: false}}]}}, {$group: {_id: "$tradingPair.symbol", sum: {$sum: "$size"}}}])

\end{lstlisting}
Výsledkem je dokument, který obsahuje následující atributy: 

\begin{itemize}
    \item \texttt{_id} -- identifikátor kryptoměny, jejíž nákupy byly sčítány
    \item \texttt{sum} -- součet všech nákupů vybrané kryptoměny
\end{itemize}


\subsection{Spojení všech transakcí s portfóliemi do kterých patří}
K spojení všech portfólií a transkací, které jsou k nim přiřazeny, slouží následující dotaz:

\begin{lstlisting}[language=JavaScript]

db.portfolios.aggregate([{$lookup: {"from": "transactions", "localField": "_id", "foreignField": "portfolioId", "as": "transactions"}}, {$project: {"portfolio_name":"$name", "transactions":"$transactions"}}])

\end{lstlisting}
Výsledkem je dokument, který obsahuje následující atributy: 

\begin{itemize}
    \item \texttt{_id} -- identifikátor portfólia
    \item \texttt{portfolio_name} -- název portfólia
    \item \texttt{transactions} -- objekt obsahující kolekci dokumentů představujících transakce
\end{itemize}

\section{Diskuze nad řešením v dané technologii}
k
Jednotlivé soubory představují exportované informace o nákupech z různých burz. Problematické je to, že na jednom místě jsou definované všechny informace o portfóliu (název a popis), avšak v dalších XML souborech, ve kterých se nachází další nákupy patřící do daného portfólia, je toto portfólio referencováno stejným prvkem, avšak atributy pro specifikaci názvu a popisu jsou vynechány. Portfólio je referencováno pomocí jeho identifikátoru. XML databáze pak pomalu degraduje na relační databázi a mohou vznikat konflikty, kdy jedno portfólio bude mít název a popis specifikovaný na různých místech.

Struktura řešení je navržena tak, že identifikátory, reference a popisy portfólií se nacházejí v atributech. 

Datový formát pro uložení dat z této domény je vhodný spíše pro jednorázové zpracování a následné uložení do jiného formátu, nikoliv jako pro implementaci hlavního úložiště.

Limitace dotazování spočívá v tom, že obchodovací páry v jednotlivých obchodech jsou referencovány přes ID, a tak není možné v rámci jednoho dotazu získat obchody včetně příslušných obchodních párů.

Mezi vhodná metadata při ukládání do relační databáze patří datum exportu z burzy a zdroj, ze kterého export pochází.

Jako jediný nepovinný údaj se v databázové struktuře používá atribut \texttt{sell}, který označuje prodeje. Jeho přítomnost není problematická.



\end{document} 
